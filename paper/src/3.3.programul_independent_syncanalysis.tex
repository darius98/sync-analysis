\subsection{Programul de analiză post-mortem SyncAnalysis}
\label{executable}

Programul \lstinline{SyncAnalysis} este a doua componentă a sistemului
de creare de unelte. Acesta se ocupă de a citi evenimentele din fișierul
DUMP scris de biblioteca de capturare de evenimente și a le analiza prin
intermediul unor analizori scriși de client.

Acest program are ca scop rezolvarea unor probleme ce fac scrierea de
unelte de analiză \textit{post-mortem} dificilă: parsarea în mod
eficient a evenimentelor capturate în timpul execuției, acces ușor la
informații de interes (cum ar fi indentitatea obiectelor țintă a
evenimentelor) și scrierea de rapoarte ce conțin toate informațiile
necesare pentru rezolvarea problemelor descoperite prin analiză. Prin
folosirea acestui program, scrierea unui nou \textit{analizor}
(componentă a unei unelte de analiză ce diagnostichează un anumit tip de
problemă) trebuie să se rezume la a implementa strict algoritmul de
diagnosticare a problemei și nimic altceva. În secțiunea
\textbf{\ref{analyzers}} se pot vedea câteva exemple de astfel de
analizori ce rezolvă probleme reale, de multe ori implementate în mai
puțin de 100 de linii de cod.

\subsubsection{Argumentele pentru linia de comandă}

Programul \lstinline{SyncAnalysis} este proiectat pentru a fi invocat de
la linia de comandă:
\begin{lstlisting}
    % sync_analysis -b program sync_analysis.dump
\end{lstlisting}
Primul lucru pe care îl face programul este să interpreteze argumentele
date în comanda de invocare (în cazul de mai sus \lstinline{"-b"},
\lstinline{"program"} și \lstinline{"sync_analysis.dump"}). În general,
forma acestor argumente este următoarea:
\begin{lstlisting}
    % sync_analysis [-optiuni] calea/la/sync_analysis.dump
\end{lstlisting}
Argumentul pentru calea la fișierul \lstinline{sync_analysis.dump} este
singurul obligatoriu pentru funcționarea executabilului. Acesta indică
fișierul DUMP ce trebuie analizat. Toate celelalte argumente reprezintă
\textit{opțiuni}, și pentru a urma tiparul executabilelor preinstalate
pe sistemele GNU/Linux (de exemplu programele din
\lstinline{binutils}\cite{binutils}), acesta încep cu \lstinline{-} sau
\lstinline{--}. Pentru interpretarea argumentelor din linia de comandă
în aceste opțiuni s-a folosit biblioteca
\lstinline{mcga-cli}\cite{mcga-cli}. Aceasta oferă un mod elegant de a
defini în opțiunile sub formă de variabile în codul sursă, a căror
valoare este modificată conform argumentelor date în linia de comandă,
indiferent de ordinea în care apar acestea. De asemenea, biblioteca
generează și afișează automat o listă a tuturor opțiunilor pe care le
poate primi programul și descrieri ale acestora într-un format ușor de
citit prin invocarea programului cu un singur argument,
\lstinline{--help} sau \lstinline{-h}.

Cea mai importantă dintre opțiuni este
\lstinline{--binary=calea/programului/client} (sau varianta mai
scurtă \lstinline{-b calea/programului/client}), folosită pentru a
indica fișierul executabil prin a cărui execuție s-a generat fișierului
\lstinline{sync_analysis.dump} analizat. Această opțiune este necesară
pentru a reconstrui lista simbolurilor funcțiilor din stiva de execuție
a capturării unui eveniment dintr-un raport. Detalii despre cum se face
această reconstrucție se găsesc în secțiunea
\textbf{\ref{stack-reconstruction}}.

O altă opțiune importantă pentru clienții ce vor să invoce programul în
procedeele lor de testare și integrare automată este
\lstinline{--report-file} (sau varianta scurtă \lstinline{-r}).
Folosirea acestei opțiuni face ca rapoartele să fie scrise în fișierul
specificat în loc să fie afișate în consolă (cum se întâmplă în mod
implicit).

Opțiunile \lstinline{--debug} și \lstinline{--print-header} controlează
afișarea unor informații în plus despre execuția programului
\lstinline{SyncAnalysis} în sine. Sunt opțiuni utile pentru
dezvolatatorii de analizori, care pot astfel observa greșeli în modul
cum au programat sau integrat analizorii în program, sau pot chiar
observa și semnala probleme cu funcționarea programului propriu-zis.

\subsubsection{Înregistrarea analizorilor}

După interpretarea opțiunilor, programul încarcă analizorii, urmând
indicațiile date de opțiunile \lstinline{-A} și \lstinline{-a}. Opțiunea
\lstinline{-A} este folosită pentru a specifica directoare în care să se
caute analizori. Un fișier găsit în oricare din directoarele specificate
prin această opțiune este considerat un \textit{analizor} dacă respectă
următoarele condiții:
\begin{itemize}
    \item Numele fișierului începe cu prefixul
    \lstinline{libsyan-analyzer-} și are extensia \lstinline{.so} (sau
    \lstinline{.dylib} pe Mac OS);
    \item Fișierul poate fi încărcat cu succes printr-un apel la funcția
    \lstinline{dlopen} (este un fișier în formatul ELF\cite{ELF} ce
    reprezintă o bibliotecă dinamică);
    \item Simbolul \lstinline{"syan_analyzer"} poate fi găsit în
    biblioteca reprezentată în fișier printr-un apel la funcția
    \lstinline{dlsym} și simbolul este asociat unui șir de caractere;
\end{itemize}
Șirul de caractere asociat simbolului \lstinline{"syan_analyzer"} este
numele analizorului. Acest nume este folosit pentru a stabili dacă
analizorul este folosit sau nu. Opțiunea \lstinline{-a} specifiă reguli
pentru ce nume trebuie să aibă un analizor pentru a fi folosit sau
ignorat: \textit{ultima} regulă dintre cele date prin opțiunea
\lstinline{-a} pe care o respectă numele unui analizor trebuie să fie
una \textit{pozitivă} pentru ca analizorul să fie folosit. O regulă este
un \textit{glob} ce conține doar litere, cifre, \lstinline{-},
\lstinline{_} sau \lstinline{*}. Numele unui analizor \textit{respectă}
regula dacă aparițiile caracterului \lstinline{*} din regulă pot fi
înlocuite prin secvențe de alte caractere permise astfel încât să se
obțină numele analizorului (de exemplu, numele
\lstinline{"redundant-recursive-mutex"} respectă regulile
\lstinline{"redundant-*"} sau \lstinline{"*mutex"}, dar nu
\lstinline{"rwlock-*"}). O regulă este \textit{negativă} dacă primul său
caracter este \lstinline{-} (și atunci primul caracter \lstinline{-}
este ignorat în verificarea regulii), altfel este \textit{pozitivă}.

Acest mod de a filtra analizorii pe bază de listă de reguli
\textit{glob} este inspirat din modul de folosire al
Clang-Tidy\cite{ClangTidy}, care folosește mai mult sau mai puțin
același algoritm de filtrare pe baza unui argument numit
\lstinline{-checks}.

\subsubsection{Procedeul de analiză}

După încărcarea analizorilor, începe procedeul de analiză propriu-zis.
Mai întâi se execută codul de inițializare înregistrat de fiecare
analizor. Reamintim aici că analizorii sunt încărcați de program ca
biblioteci dinamice prin apeluri manuale la funcția \lstinline{dlopen}
din biblioteca de sistem \lstinline{DL}\cite{DL}. Dacă biblioteca unui
analizor expune public simbolul \lstinline{"syan_analyzer_start_up"},
acesta trebuie să fie asociat unei funcții ce nu primește parametri, și
această funcție este apelată acum, la începutul analizei.

După inițializarea analizorilor, programul citește evenimentele din
fișierul DUMP și le expune unul câte unul analizorilor, în ordinea din
fișier. Următoarea funcție este disponibilă analizorilor:
\begin{lstlisting}
    namespace syan { syan::Event current_event(); }
\end{lstlisting}
Pentru fiecare eveniment, acesta este pus într-o variabilă globală ce
poate fi accesată de analizori prin funcția de mai sus. Apoi, pentru
fiecare analizor a cărui bibliotecă expune public simbolul
\lstinline{"syan_analyzer_on_event"}, acesta trebuie să fie asociat unei
funcții ce nu primește parametri, și această funcție este apelată.

După ce tot fluxul de evenimente din fișierul DUMP este epuizat, se
caută în biblioteca fiecărui analizor simbolul
\lstinline{"syan_analyzer_shut_down"}, care trebuie din nou să fie
asociat unei funcții ce nu primește parametri. Pentru analizorii unde
acest simbol există, funcția asociată este apelată acum, după
parcurgerea tuturor evenimentelor.

\subsubsection{Clasa \lstinline{syan::Event}}

Am menționat deja că un analizor are acces la evenimentul curent
procesat prin funcția \lstinline{current_event()} care returnează o
instanță a clasei \lstinline{syan::Event}. Această clasă oferă
informațiile despre evenimentul respectiv prin câteva metode simple:
\begin{lstlisting}
    int type() const;
    ObjectId object() const;
    ObjectId thread() const;
    int64_t time_rel_to_program_start() const;
\end{lstlisting}
Aceste metode expun exact câmpurile ce au fost serializate în fișierul
DUMP.

O instanță a acestei clase poate fi copiată de un analizor dacă
acesta are nevoie de informația din acest eveniment pentru un posibil
raport viitor. Pentru că un eveniment ocupă suficient de multă memorie
ca să fie ineficient de copiat de mai mulți analizori, informația în
sine despre eveniment este stocată o singură dată, iar clasa
\lstinline{syan::Event} acționează similar cu un
\textit{shared pointer}\cite{SharedPtr}: de fiecare dată când o instanță
este copiată, se incrementează un contor pentru numărul de referințe,
iar de fiecare dată când o instanță este distrusă este decrementat acest
contor. Când contorul ajunge la 0, memoria asociată unicei instanțe a
informației propriu-zise a evenimentului este dealocată. Copierea unui
\lstinline{syan::Event} este deci eficientă, și mai mulți analizori pot
păstra referințe către un eveniment fără a avea un impact negativ asupra
performanței analizei.

\subsubsection{Crearea unui raport}

Scopul principal al analizorilor este să găsească probleme prin
observarea evenimentelor capturate. Când un analizor găsește o astfel de
problemă, acesta trebuie să informeze programul că trebuie creat un
raport pentru problema respectivă. Următoarea funcție este disponibilă
analizorilor:
\begin{lstlisting}
    namespace syan { syan::Report create_report(); }
\end{lstlisting}
Instanța clasei \lstinline{syan::Report} astfel obținută controlează
trimiterea unui raport către utilizator. Această clasă expune public
următoarele 4 metode:
\begin{lstlisting}
    Report& set_level(Level level);
    Report& set_description(std::string description);
    Report& add_section(std::string description, Event event);
    void send();
\end{lstlisting}
Un raport are un \textit{nivel de severitate}, o \textit{descriere} și
zero sau mai multe \textit{secțiuni}. Nivelul de severitate poate fi
\textit{informație} (\lstinline{Level::info}), \textit{avertisment}
(\lstinline{Level::warning}) sau \textit{eroare}
(\lstinline{Level::error}), oferind o indicație simplă pentru tipul de
diagnostic emis. În general se recomandă ca nivelul de \textit{eroare}
să fie folosit pentru diagnostice care mereu indică o greșeală ce
trebuie rezolvată în program, nivelul de \textit{avertisment} să fie
folosit pentru diagnostice care \textit{cel mai probabil} indică o
problemă, dar pot fi totuși incorecte în anumite situații și nivelul de
\textit{informație} pentru orice alt tip de diagnostic.

\textit{Descrierea} este mesajul principal al raportului: aceasta spune
ce problemă a fost descoperită de analiză.

\textit{Secțiunile} unui raport sunt atașamente la mesajul principal dat
de \textit{descriere}. O \textit{secțiune} conține un eveniment și o
descriere. Descrierea ar trebui să explice ce rol joacă acest eveniment
în diagnostic sau de ce este important. Evenimentul este inclus pentru
ca această secțiune a raportului să conțină momentul de timp când a avut
loc acest eveniment (atât în valoare absolută cât și relativ la
începutul programului), firul de execuție pe care s-a capturat
evenimentul și funcțiile din stiva de execuție a capturării
evenimentului (vezi Secțiunea \textbf{\ref{stack-reconstruction}}).

După construirea unui raport, acesta poate fi trimis manual folosind
metoda \lstinline{.send()}. Dacă nu se face apel la metoda
\lstinline{.send()} pentru un raport, acesta va apela automat metoda în
destructor.

\subsubsection{Baza de date a obiectelor active}\label{obj-database}

Pe baza evenimentelor ce conțin în câmpul de tip
\textit{biții indicatori} \lstinline{SA_EV_CREATE} și
\lstinline{SA_EV_DESTROY} (vezi secțiunea
\textbf{\ref{section:library-interface}}), programul menține
(independent de analizori) o bază de date internă dar ușor accesibilă de
obiecte \textit{active} (care au fost create dar nu distruse), incluzând
o listă cu toate firele de execuție create (acestea nu se consideră
niciodată distruse). Pentru fiecare din aceste obiecte este stocat
evenimentul care a creat obiectul. De asemenea unui obiect îi este
asociat un identificator unic numeric (independent de cel din câmpul de
\lstinline{object_id}), pentru a putea fi folosit în rapoarte ca
\textit{numele} obiectului (fără riscul de a apărea coliziuni, risc ce
există pentru \lstinline{object_id}, după cum este descris în secțiunea
\textbf{\ref{section:library-interface}}).

Cu ajutorul acestei baze de date, pentru orice secțiune adăugată unui
raport, se mai adaugă automat până la două alte secțiuni: evenimentul ce
a creat obiectul țintă al evenimentului din secțiune (dacă există în
baza de date) și evenimentul ce a creat firul de execuție al
evenimentului din secțiune (dacă există în baza de date).

Baza de date este accesibilă analizorilor prin următoarea funcție:
\begin{lstlisting}
    namespace syan { const syan::Database& database(); }
\end{lstlisting}
Analizorii se pot folosi de baza de date pentru a avea acces programatic
la informațiile menționate mai sus, pentru care clasa
\lstinline{syan::Database} oferă metode de acces.

\subsubsection{Afișarea simbolurilor din stiva de execuție}
\label{stack-reconstruction}

\todo{Write text}

\subsubsection{Interfața C++ pentru un analizor}

Se recomandă scrierea analizorilor în limbajul C++, pentru că funcțiile
expuse de program pentru analizori (\lstinline{current_event},
\lstinline{database} și \lstinline{create_report}) respectiv clasele a
căror instanțe sunt întoarse de aceste funcții (\lstinline{Event},
\lstinline{Database} și \lstinline{Report}) sunt interfețe C++.

Un mod rudimentar de a scrie un analizor în C++ ce expune toate cele 4
simboluri căutate de program este văzut în Fragmentul de cod
\ref{code:analyzer-C}.
\begin{lstlisting}[caption=Interfața rudimentară a unui analizor,
                   label=code:analyzer-C]
    extern "C" const char* syan_analyzer = "analizorul-meu";
    extern "C" void syan_analyzer_start_up() {
        // cod pentru initializare
    }
    extern "C" void syan_analyzer_on_event() {
        const syan::Event event = syan::current_event();
        // cod pentru procesarea unui eveniment
    }
    extern "C" void syan_analyzer_shut_down() {
        // cod pentru oprire
    }
\end{lstlisting}
Motivul pentru care simbolurile pe care le caută programul în
bibliotecile analizorilor sunt simboluri C și pentru care funcțiile ce
trebuie asociate acestor simboluri au signatură C este pentru că
interfața binară a limbajului C++ nu este standardizată oficial de ISO,
pentrum cea de C.

Această interfață nu este cea mai elegantă pentru limbajul C++.
Interfața de mai sus încurajează abuzul de variabile globale,
încurajează eliberarea manuală a memoriei și alte practici periculoase
pentru care C++ oferă soluții mai bune. Așadar, în interfața expusă
analizorilor este oferit și un \textit{macro} ce are ca scop construirea
tuturor celor 4 simboluri necesare pe baza unei clase. Clasa trebuie să
expună public un constructor fără niciun argument (din care se va genera
simbolul \lstinline{"syan_analyzer_start_up"}), un destructor (din care
se va genera simbolul \lstinline{"syan_analyzer_shut_down"}), o
variabilă membru statică cu tipul de date \lstinline{const char*} și
numele \lstinline{name} (din care se va genera simbolul
\lstinline{"syan_analyzer"}) și o suprascriere a operatorului
\lstinline{()} ce primește ca unic parametru un obiect de tipul
\lstinline{syan::Event} (din care se va genera simbolul
\lstinline{"syan_analyzer_on_event"}). Un exemplu de folosire al acestui
\textit{macro} poate fi văzut în Fragmentul de cod
\ref{code:analyzer-CXX}.

\begin{lstlisting}[caption=Interfața pe bază de clasă a unui analizor,
                   label=code:analyzer-CXX]
    struct AnalizorulMeu {
        static constexpr const char* name = "analizorul-meu";

        AnalizorulMeu() { /* cod pentru initializare */ }
        ~AnalizorulMeu() { /* cod pentru oprire */ }

        void operator()(Event event) {
            // cod pentru procesarea unui eveniment
        }
    };
    SYAN_ANALYZER_CLASS(AnalizorulMeu);
\end{lstlisting}
