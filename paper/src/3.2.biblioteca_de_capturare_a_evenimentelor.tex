\subsection{Biblioteca de capturare a evenimentelor}
Aceasta este prima și cea mai importantă dintre componentele
proiectului. Pentru că această bibliotecă este încărcată direct în
programul clientului, este proiectată să fie atât portabilă cât și
performantă.

Pentru a maximiza portabilitatea, alegerea naturală de limbaj
pentru bibliotecă este C\cite{C}: pentru că limbajul are o interfață
binară standardizată de ISO și implementată de majoritatea
compilatoarelor folosite în industrie, împreună cu faptul că majoritatea
limbajelor de nivel înalt implementează
\textit{foreign-function-interface} cu C, aplicațiile pot beneficia de
această bibliotecă aproape indiferent de limbajul în care au fost
programate sau platforma pe care sunt executate.

Deși există variante mai bune pentru performanță cum ar fi C++ sau
\textit{assembly}, C este un limbaj suficient de eficient pentru a fi
cea mai bună alegere, luând în considerare avantajul copleșitor de
portabilitate.

\subsubsection{Interfață}
Pentru că această bibliotecă este una generică, interfața este
intenționat minimală: sunt expuse în mod public 3 funcții și o constantă
numerică. În limbajul C, interfața poate fi vizualizată astfel:
\begin{lstlisting}[
        language=C,
        captionpos=b,
        caption=Interfața bibliotecii pentru capturare de evenimente]
    void syan_capture_event(int event_type, void* object);

    #define SYAN_THREAD_CREATE_EVENT 17

    void* syan_initialize_event(int event_type);
    void syan_finalize_event(void* event, void* object);
\end{lstlisting}
Funcționalitatea a acestei biblioteci este de a captura evenimente, așa
că interfața publică importantă este doar funcția
\lstinline{syan_capture_event}, care îndeplinește exact acest scop.
Exemplul principal de utilizare după care a fost proiectată funcția
\lstinline{syan_capture_event} este următorul:
\begin{lstlisting}[caption=Exemplul 1 folosit în proiectarea interfeței]
    #define EVENT_MUTEX_BEFORE_LOCK 1331
    #define EVENT_MUTEX_AFTER_LOCK 1332

    pthread_mutex_t* m;
    void f() {
        syan_capture_event(EVENT_MUTEX_BEFORE_LOCK, m);
        pthread_mutex_lock(m);
        syan_capture_event(EVENT_MUTEX_AFTER_LOCK, m);
    }
\end{lstlisting}
Funcția este apelată cu 2 parametri, unul pentru tipul de eveniment
capturat și celălalt pentru identitatea obiectului țintă al
evenimentului. Astfel, în analiza post-mortem a unui program se poate
face o corelare între mai multe evenimente care au avut loc pe același
obiect țintă. Cum obiectele în limbajul C au în mod implicit o
identitate dată de adresa de memorie unde sunt stocate, este natural ca
parametrul care identifică obiectul să aibă tipul de date
\lstinline{void*}, care reprezintă în mod convențional o adresă oarecare
de memorie. Apare totuși o problemă: aceeași adresă de memorie poate fi
refolosită de-a lungul execuției unui program, pentru a stoca mai multe
obiecte logice diferite (după dealocarea primului obiect, la alocarea
unui obiect nou se poate folosi aceeași memorie; un exemplu comun este
memoria stivei de execuție). Pentru a evita confuzii generate de această
problemă, se recomandă ca pentru un obiect ce este ținta unor evenimente
capturate să se captureze și un eveniment care marchează dealocarea
obiectului.

Fiecare eveniment capturat conține identitatea firului de execuție care
l-a capturat. Este util atunci să se raporteze câte un eveniment care
marchează crearea fiecărui fir de execuție, pentru a putea pune
evenimentele unui fir de execuție într-un context. Astfel, biblioteca în
sine când se încarcă raportează ca prim eveniment unul care marchează
crearea firului de execuție care a încărcat biblioteca. Pentru că
biblioteca raportează în interiorul său un eveniment de creare de fir de
execuție, aceasta stabilește un număr
\lstinline{SYAN_THREAD_CREATE_EVENT} (aleator ales 17) care să
reprezinte evenimentele de acest tip. Orice bibliotecă de integrare
trebuie să dea același număr pentru argumentul \lstinline{event_type}
când apelează funcția \lstinline{syan_capture_event} pentru a raporta
crearea unui fir de execuție.

Cu experiența implementării bibliotecilor de integrare s-a observat că
atunci când se raportează crearea unui nou fir de execuție folosind
biblioteca \lstinline{pthread}, ținta evenimentului este cunoscută abia
după ce a pornit firul de execuție:
\begin{lstlisting}[caption=Exemplul 2 folosit în proiectarea interfeței]
    pthread_t T;
    pthread_create(&T, nullptr, func, arg);
    syan_capture_event(SYAN_THREAD_CREATE_EVENT, (void*)T);
\end{lstlisting}
Dacă evenimentul care marchează crearea firului de execuție \textit{T}
este raportat \textit{după} crearea lui \textit{T}, pot apărea
inconsistențe în analiza post-mortem: firul de execuție \textit{T} poate
începe să captureze evenimente, care pot ajunge să apară în fișierul de
evenimente înaintea evenimentului care marchează crearea lui \textit{T}.
Pentru a rezolva această situație, biblioteca expune celelalte două
funcții \lstinline{syan_initialize_event} și
\lstinline{syan_finalize_event} în interfața bibliotecii. Exemplul de
mai sus se repară astfel:
\begin{lstlisting}[caption=Exemplul 3 folosit în proiectarea interfeței]
    pthread_t T;
    void* ev = syan_initialize_event(SYAN_THREAD_CREATE_EVENT);
    pthread_create(&T, nullptr, func, arg);
    syan_finalize_event(ev, (void*)T);
\end{lstlisting}
Funcția \lstinline{syan_initialize_event} crează evenimentul și îi
alocă locul în fișierul de evenimente \textit{înainte} de crearea
firului \textit{T}. După crearea firului \textit{T}, evenimentul este
doar marcat ca fiind \textit{finalizat}, dar este scris în fișier ca și
cum în locul unde a fost mai înainte apelat
\lstinline{syan_initialize_event} ar fi fost apelat
\lstinline{syan_capture_event}, rezolvând problema ridicată mai devreme.

Se observă acum că funcția \lstinline{syan_capture_event} are o
implementare evidentă pe baza celor două funcții nou introduse:
\begin{lstlisting}[caption=Implementarea funcției
                   \lstinline{syan_capture_event}]
    void syan_capture_event(int event_type, void* object) {
        void* ev = syan_initialize_event(event_type);
        syan_finalize_event(ev, object);
    }
\end{lstlisting}

\subsubsection{Structura unui eveniment}

\subsubsection{Obținerea stivei de execuție}

\subsubsection{Coada de evenimente}

\subsubsection{Serializarea fișiserului DUMP}
