\subsection{Tipuri de evenimente pentru primitive de sincronizare}
\label{section:event-types}

Astfel, am creat
o interfață C ce conține constante numerice pentru tipurile de
evenimente specifice primitivelor de sincronizare, ce complimentează
constantele numerice expuse în interfața bibliotecii pentru capturarea
de evenimente.

Definim mai întâi câteva constante ce reprezintă
\textit{biți indicatori} din tipul evenimentului: tipul de obiect
țintă al evenimentului (\lstinline{SA_MUTEX} pentru \textit{mutex},
\lstinline{SA_REC_MUTEX} pentru \textit{mutex recursiv} și
\lstinline{SA_RWLOCK} pentru \textit{read-write lock}, similare cu
\lstinline{SA_EV_THREAD} pentru \textit{fire de execuție}, definită în
Secțiunea \textbf{\ref{section:library-interface}}) și acțiunea ce are
loc pe acel obiect țintă (\lstinline{SA_BEFORE_WR_LOCK},
\lstinline{SA_AFTER_WR_LOCK} și altele, din nou similare cu constantele
\lstinline{SA_EV_CREATE} și \lstinline{SA_EV_DESTROY} din Secțiunea
\textbf{\ref{section:library-interface}}). Lista completă a tipurilor de
evenimente capturate, împreună cu valorile acestora obținute din biții
indicatori este următoarea:

\begin{lstlisting}[caption=Tipurile de evenimente pentru primitivele de
                           sincronizare, label=code:syan-event-types]
SA_EV_THREAD_ON_JOIN        = SA_EV_THREAD | SA_JOIN
SA_EV_THREAD_ON_DETACH      = SA_EV_THREAD | SA_DETACH
SA_EV_MUTEX_ON_CREATE       = SA_MUTEX     | SA_EV_CREATE
SA_EV_MUTEX_ON_TRY_LOCK     = SA_MUTEX     | SA_TRY_WR_LOCK
SA_EV_MUTEX_BEFORE_LOCK     = SA_MUTEX     | SA_BEFORE_WR_LOCK
SA_EV_MUTEX_AFTER_LOCK      = SA_MUTEX     | SA_AFTER_WR_LOCK
SA_EV_MUTEX_ON_UNLOCK       = SA_MUTEX     | SA_WR_UNLOCK
SA_EV_MUTEX_ON_DESTROY      = SA_MUTEX     | SA_EV_DESTROY
SA_EV_REC_MUTEX_ON_CREATE   = SA_REC_MUTEX | SA_EV_CREATE
SA_EV_RWLOCK_ON_CREATE      = SA_RWLOCK    | SA_EV_CREATE
SA_EV_RWLOCK_ON_TRY_RD_LOCK = SA_RWLOCK    | SA_TRY_RD_LOCK
SA_EV_RWLOCK_BEFORE_RD_LOCK = SA_RWLOCK    | SA_BEFORE_RD_LOCK
SA_EV_RWLOCK_AFTER_RD_LOCK  = SA_RWLOCK    | SA_AFTER_RD_LOCK
SA_EV_RWLOCK_ON_TRY_WR_LOCK = SA_RWLOCK    | SA_TRY_WR_LOCK
SA_EV_RWLOCK_BEFORE_WR_LOCK = SA_RWLOCK    | SA_BEFORE_WR_LOCK
SA_EV_RWLOCK_AFTER_WR_LOCK  = SA_RWLOCK    | SA_AFTER_WR_LOCK
SA_EV_RWLOCK_ON_UNLOCK      = SA_RWLOCK    | SA_WR_UNLOCK
SA_EV_RWLOCK_ON_DESTROY     = SA_RWLOCK    | SA_EV_DESTROY
\end{lstlisting}

Pentru primitiva \textit{mutex recursiv}, se vor captura aceleași
evenimente ca și pentru primitiva \textit{mutex}, mai puțin la crearea
obiectului (\lstinline{SA_EV_MUTEX_ON_CREATE} versus
\lstinline{SA_EV_REC_MUTEX_ON_CREATE}). Analizorii executați post-mortem
vor putea observa dacă un eveniment comun (de exemplu
\lstinline{SA_EV_MUTEX_AFTER_LOCK}) a avut loc pe un \textit{mutex} sau
pe un \textit{mutex recursiv} pe baza evenimentului ce a creat obiectul
respectiv, eveniment ce este stocat în baza de date a programului
\lstinline{SyncAnalysis} descrisă în secțiunea
\textbf{\ref{obj-database}}.
