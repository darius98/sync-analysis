\subsection{Biblioteci de integrare}
\label{integration-libraries}

\subsubsection{cxxsync (C++, începând cu C++11)}
\lstinline{cxxsync} este o bibliotecă de C++ proiectată pentru a fi un
exemplu de integrare cu biblioteca de capturare de evenimente. Nu este
neapărat cea mai bună soluție pentru integrarea în proiecte existente,
deoarece are o interfață diferită de alte biblioteci existente, deci
sunt necesare schimbări în codul sursă pentru integrare.

Biblioteca expune 4 clase principale, pentru cele 4 primitive ale căror
evenimente ne interesează: \lstinline{Thread} pentru
\textit{fir de execuție}, \lstinline{Mutex} pentru \textit{mutex},
\lstinline{RecursiveMutex} pentru \textit{mutex recursiv} și
\lstinline{RWLock} pentru \textit{read-write lock}. În implementarea
tuturor claselor se folosesc funcții din biblioteca \lstinline{pthread}
și se capturează evenimente folosind funcția
\lstinline{syan_capture_event}, descrisă în Secțiunea
\textbf{\ref{section:library-interface}}, și tipuri de evenimente dintre
cele descrise în Fragmentul de cod \ref{code:syan-event-types}.

Clasa \lstinline{Thread} crează un fir de execuție printr-un apel al
funcției \lstinline{pthread_create} în constructor. Pentru a admite
sintaxa elegantă de funcții \textit{lambda} din C++11 pentru
inițializarea firului de execuție, s-a folosit tehnica C++ cunoscută ca
\textit{type erasure}\cite{TypeErasure}, mai exact transformarea unui
tip de date necunoscut, primit ca \textit{parametru de template},
într-unul cunoscut, în acest caz \lstinline{void*(*)(void*)}
(\textit{pointer} la o funcție ce primește ca parametru un
\lstinline{void*} și returnează un \lstinline{void*}; acesta este tipul
de date pe care îl primește funcția \lstinline{pthread_create} ca
parametru pentru a inițializa noul fir de execuție). Implementarea
constructorului clasei \lstinline{Thread} folosind această tehnică,
și de asemenea folosirea funcțiilor \lstinline{syan_initialize_event} și
\lstinline{syan_finalize_event} în locul funcției
\lstinline{syan_capture_event} din motivele precizate în secțiunea
\textbf{\ref{section:library-interface}} se pot vedea în Fragmentul de
cod \ref{code:thread-type-erasure}.
\begin{lstlisting}[caption=Tehnica \textit{type-erasure} în
                           clasa \lstinline{Thread} din
                           \lstinline{cxxsync},
                   label=code:thread-type-erasure]
    template<class Lambda> Thread::Thread(Lambda lambda) {
        init(&erased_lambda<Lambda>, new Lambda(lambda));
    }

    void Thread::init(void* (*func)(void*), void* arg) {
        auto e = syan_initialize_event(SA_EV_THREAD_ON_CREATE);
        pthread_create(&pt_thread, nullptr, func, arg);
        syan_finalize_event(e, (void*)pt_thread);
    }

    template<class Lambda> void* erased_lambda(void* arg) {
        Lambda* lambda = static_cast<Lambda*>(arg);
        (*lambda)();
        delete lambda;
        return nullptr;
    }
\end{lstlisting}

Clasele \lstinline{Mutex} și \lstinline{RecursiveMutex} expun fiecare
câte 3 metode publice:
\begin{lstlisting}
    bool try_lock() noexcept;
    void lock();
    void unlock();
\end{lstlisting}

Aceste trei metode sunt implementate la fel în ambele clase, apelând în
interior funcțiile \lstinline{pthread_mutex_trylock},
\lstinline{_lock} respectiv \lstinline{_unlock} din biblioteca
\lstinline{pthread}. Clasa \lstinline{RWLock} expune 6 metode publice
(3 pentru \textit{read-lock} și 3 pentru \textit{write-lock}):

\begin{lstlisting}
    bool try_rd_lock() noexcept;
    void rd_lock();
    void rd_unlock();
    bool try_wr_lock() noexcept;
    void wr_lock();
    void wr_unlock();
\end{lstlisting}
Metodele acestea sunt implementate prin apeluri la funcțiile
echivalente din \lstinline{pthread}:
\lstinline{pthread_rwlock_tryrdlock}, \lstinline{_rdlock},
\lstinline{_unlock}, \lstinline{_trywrlock} și \lstinline{_wrlock}.

Toate cele 4 clase principale din bibliotecă fac unul sau mai multe
apeluri la funcția \lstinline{syan_capture_event}, pentru a raporta
diverse evenimente. Spre exemplu, clasa \lstinline{RecursiveMutex} face
în constructor următorul apel:
\begin{lstlisting}
    syan_capture_event(SA_EV_REC_MUTEX_ON_CREATE, this);
\end{lstlisting}
Metodele \lstinline{try_lock, try_rd_lock, try_wr_lock} pot face unul
sau două apeluri, în funcție de dacă primitiva a fost \textit{locked} cu
succes:
\begin{lstlisting}[caption=Metoda \lstinline{RWLock::try_rd_lock} din
                           \lstinline{cxxsync}]
    bool RWLock::try_rd_lock() noexcept {
      syan_capture_event(SA_EV_RWLOCK_ON_TRY_RD_LOCK, this);
      if (pthread_rwlock_tryrdlock(&pt_rwlock) != 0)
        return false;
      syan_capture_event(SA_EV_RWLOCK_AFTER_RD_LOCK, this);
      return true;
    }
\end{lstlisting}

Pe lângă cele 4 clase principale, biblioteca mai expune și 4 clase
auxiliare atașate pentru a implementa tehnica \textit{RAII}\cite{RAII},
foarte populară și utilă în programele moderne C++ (clasele sunt numite
\lstinline{MutexGuard}, \lstinline{RecursiveMutexGuard},
\lstinline{RWLockReadGuard} și \lstinline{RWLockWriteGuard}). Aceste
clase au o interfață foarte simplă (compusă doar din constructor și
destructor) și o implementare la fel de simplă, expusă în Fragmentul de
cod \ref{code:mutex-guard}. Scopul acestor clase este doar ca o instanță
cu \textit{automatic storage} să fie creată unde se vrea un
\textit{lock} pe o primitivă și metoda asociată de \textit{unlock} se va
apela automat la sfârșitul blocului respectiv de cod (indiferent dacă se
iese în mod natural prin terminarea blocului sau printr-un
\lstinline{throw}, \lstinline{goto}, \lstinline{break},
\lstinline{continue} sau \lstinline{return}).
\begin{lstlisting}[caption=Clasa \lstinline{MutexGuard} din
                           \lstinline{cxxsync},
                   label=code:mutex-guard]
    struct MutexGuard {
        explicit MutexGuard(Mutex* m): mtx(m) { mtx->lock(); }
        ~MutexGuard() { mtx->unlock(); }
    private:
        Mutex* mtx;
    }
\end{lstlisting}

\subsubsection{pthread shim (C/C++)}
Multe programe ce folosesc mai multe fire de execuție se folosesc de
interfața bibliotecii \lstinline{pthread} pentru a crea și sincroniza
aceste fire, pentru că aceasta este disponibilă pe orice sistem de
operare ce respectă interfața POSIX.

Biblioteca \lstinline{syan_pthread_shim} este o bibliotecă dinamică ce
exportă o parte din simbolurile disponibile în biblioteca
\lstinline{pthread}. Toate aceste simboluri sunt funcții care
\textit{interceptează} apelurile programului la funcții din
\lstinline{pthread}, capturează evenimentele asociate respectivelor
funcții prin apeluri la \lstinline{syan_capture_event} și apelează mai
departe funcțiile cu același simbol din \lstinline{pthread}. Această
bibliotecă nu oferă o interfață în cod sursă, pentru că interfața la
nivel de sursă este o submulțime a funcțiilor din interfața
\lstinline{pthread}. Precum orice alt \textit{shim} (bibliotecă ce
interceptează în mod transparent apelurile la altă bibliotecă
\cite{wiki:Shim_(computing)}), această bibliotecă trebuie încărcată de
un program ce folosește \lstinline{pthread} \textit{înainte} de
încărcarea bibliotecii \lstinline{pthread} propriu-zisă, pentru ca
apelurile la funcții cu simbolurile vizibile din \lstinline{pthread} să
fie interpretate ca apeluri la funcții din biblioteca
\lstinline{syan_pthread_shim}.

Pentru a folosi această bibliotecă, ne vom folosi de bine-cunoscutul
\textit{LD\_PRELOAD trick}\cite{LD_PRELOAD} pentru a garanta că
biblioteca este încărcată prima. Trucul se bazează pe variabila de mediu
\lstinline{LD_PRELOAD}: dacă această variabilă conține o listă de nume
de biblioteci dinamice, atunci aceste biblioteci sunt încărcate înaintea
tuturor celorlalte cerute de program:
\begin{lstlisting}
    % LD_PRELOAD="libsyan_pthread_shim.so" ./program
\end{lstlisting}
Folosind acest truc, această bibliotecă poate fi folosită fără a
necesita măcar recompilarea programului client. Când programul face
un apel la o funcție din \lstinline{pthread}, spre exemplu la funcția
\lstinline{pthread_rwlock_tryrdlock}, în loc să se apeleze funcția cu
acel nume din biblioteca \lstinline{pthread}, se va apela funcția din
biblioteca noastră (Fragmentul de cod \ref{code:pthread_shim_func}).
Aceasta face apelul mai departe la funcția originală, găsită printr-un
apel manual al funcției \lstinline{dlsym} cu parametrul
\lstinline{RTLD_NEXT} (parte din interfața DL\cite{DL}), însă capturează
de asemenea evenimentele interesante prin apeluri la funcția
\lstinline{syan_capture_event}.

\begin{lstlisting}[caption=Funcția
                           \lstinline{pthread_rwlock_tryrdlock} din
                           \lstinline{pthread_shim},
                   float, floatplacement=H,
                   label=code:pthread_shim_func]
    typedef int(*original_func_t)(pthread_rwlock_t*);
    int pthread_rwlock_tryrdlock(pthread_rwlock_t* rwl) {
      original_func_t f = (original_func_t)dlsym(
            RTLD_NEXT, "pthread_rwlock_tryrdlock");
      syan_capture_event(SA_EV_RWLOCK_ON_TRY_RD_LOCK, rwl);
      int r = f(rwl);
      if (r == 0)
        syan_capture_event(SA_EV_RWLOCK_AFTER_RD_LOCK, rwl);
      return r;
    }
\end{lstlisting}
