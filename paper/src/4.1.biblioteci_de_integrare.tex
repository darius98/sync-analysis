\subsection{Biblioteci de integrare}
\label{integration-libraries}

\subsubsection{Tipuri de evenimente pentru primitive de sincronizare}

Biblioteca pentru capturare de evenimente și programul de analiză
post-mortem sunt componente generice, refolosibile pentru orice fel de
analiză post-mortem de evenimente. Pentru a folosi aceste componente în
analiza primitivelor de sincronizare, trebuie mai întâi să definim
tipurile de evenimente ce descriu aceste primitive. Astfel, am creat
o interfață C ce conține constante numerice pentru tipurile de
evenimente specifice primitivelor de sincronizare, ce complimentează
constantele numerice oferite de interfața bibliotecii pentru capturarea
de evenimente.

Definim mai întâi câteva constante ce reprezintă
\textit{biți indicatori} din tipul evenimentului: tipul de obiect
țintă al evenimentului (\lstinline{SA_MUTEX} pentru \textit{mutex},
\lstinline{SA_REC_MUTEX} pentru \textit{mutex recursiv} și
\lstinline{SA_RWLOCK} pentru \textit{read-write lock}, similare cu
constanta \lstinline{SA_EV_THREAD} pentru \textit{fire de execuție}) și
acțiunea ce are loc pe acel obiect țintă (\lstinline{SA_BEFORE_WR_LOCK},
\lstinline{SA_AFTER_WR_LOCK} și așa mai departe, similare cu
\lstinline{SA_EV_CREATE}). Lista completă a tipurilor de evenimente
capturate, împreună cu valorile acestora obținute din biții indicatori
este următoarea:

\begin{lstlisting}[caption=Tipurile de evenimente pentru primitivele de
                           sincronizare, label=code:syan-event-types]
SA_EV_THREAD_ON_JOIN        = SA_EV_THREAD | SA_JOIN,
SA_EV_THREAD_ON_DETACH      = SA_EV_THREAD | SA_DETACH,
SA_EV_MUTEX_ON_CREATE       = SA_MUTEX     | SA_EV_CREATE,
SA_EV_MUTEX_ON_TRY_LOCK     = SA_MUTEX     | SA_TRY_WR_LOCK,
SA_EV_MUTEX_BEFORE_LOCK     = SA_MUTEX     | SA_BEFORE_WR_LOCK,
SA_EV_MUTEX_AFTER_LOCK      = SA_MUTEX     | SA_AFTER_WR_LOCK,
SA_EV_MUTEX_ON_UNLOCK       = SA_MUTEX     | SA_WR_UNLOCK,
SA_EV_MUTEX_ON_DESTROY      = SA_MUTEX     | SA_EV_DESTROY,
SA_EV_REC_MUTEX_ON_CREATE   = SA_REC_MUTEX | SA_EV_CREATE,
SA_EV_REC_MUTEX_ON_TRY_LOCK = SA_REC_MUTEX | SA_TRY_WR_LOCK,
SA_EV_REC_MUTEX_BEFORE_LOCK = SA_REC_MUTEX | SA_BEFORE_WR_LOCK,
SA_EV_REC_MUTEX_AFTER_LOCK  = SA_REC_MUTEX | SA_AFTER_WR_LOCK,
SA_EV_REC_MUTEX_ON_UNLOCK   = SA_REC_MUTEX | SA_WR_UNLOCK,
SA_EV_REC_MUTEX_ON_DESTROY  = SA_REC_MUTEX | SA_EV_DESTROY,
SA_EV_RWLOCK_ON_CREATE      = SA_RWLOCK    | SA_EV_CREATE,
SA_EV_RWLOCK_ON_TRY_RD_LOCK = SA_RWLOCK    | SA_TRY_RD_LOCK,
SA_EV_RWLOCK_BEFORE_RD_LOCK = SA_RWLOCK    | SA_BEFORE_RD_LOCK,
SA_EV_RWLOCK_AFTER_RD_LOCK  = SA_RWLOCK    | SA_AFTER_RD_LOCK,
SA_EV_RWLOCK_ON_RD_UNLOCK   = SA_RWLOCK    | SA_RD_UNLOCK,
SA_EV_RWLOCK_ON_TRY_WR_LOCK = SA_RWLOCK    | SA_TRY_WR_LOCK,
SA_EV_RWLOCK_BEFORE_WR_LOCK = SA_RWLOCK    | SA_BEFORE_WR_LOCK,
SA_EV_RWLOCK_AFTER_WR_LOCK  = SA_RWLOCK    | SA_AFTER_WR_LOCK,
SA_EV_RWLOCK_ON_WR_UNLOCK   = SA_RWLOCK    | SA_WR_UNLOCK,
SA_EV_RWLOCK_ON_DESTROY     = SA_RWLOCK    | SA_EV_DESTROY,
\end{lstlisting}

Toate cele 3 biblioteci de integrare pentru analiza primitivelor de
sincronizare descrise mai jos capturează aceste tipuri de evenimente și
folosesc exact constantele scrise în Fragmentul de cod
\ref{code:syan-event-types} pentru parametrul \lstinline{event_type} dat
funcției \lstinline{syan_capture_event}.

\subsubsection{cxxsync (C++, începând cu C++98)}
\lstinline{cxxsync} este o bibliotecă de C++ proiectată pentru a fi un
exemplu de integrare cu biblioteca de capturare de evenimente. Nu este
neapărat cea mai bună soluție pentru integrarea în proiecte existente,
deoarece are o interfață diferită de alte biblioteci existente, deci
pentru integrare sunt necesare schimbări în codul sursă. Această
bibliotecă este folosită de exemplu în testele analizorilor descriși în
secțiunea \textbf{\ref{analyzers}}.

\subsubsection{pthread shim (C/C++)}

\subsubsection{libc++ shim (pentru C++, începând cu C++11)}
