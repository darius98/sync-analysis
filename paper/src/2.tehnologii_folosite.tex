\section{Tehnologii folosite}

\subsection{Limbajul C11 și biblioteca standard C}

Limbajul C\cite{C} este folosit pentru implementarea bibliotecii de
înregistrare de evenimente și a uneia din bibliotecile de integrare.
Am ales limbajul pentru componenta aceasta din proiect deoarece
portabilitatea bibliotecii este o prioritate: această bibliotecă trebuie
să poată fi folosită în orice program scris în aproape orice limbaj și
rulând pe aproape orice sistem de operare. De asemenea este important ca
biblioteca să fie cât mai eficientă, pentru a nu interfera prea tare cu
execuția programului client. Această combinație de cerințe este
îndeplinită perfect de limbajul C. Am ales versiunea C11 pentru a avea
acces la modulul \lstinline{<stdatomic.h>} din biblioteca standard,
folosit în implementarea bibliotecii de înregistrare de evenimente din
motive de performanță.

\subsection{Limbajul C++17 și biblioteca standard C++}

Limbajul C++\cite{CXX} este folosit pentru implementarea majorității
proiectului. Programul \lstinline{SyncAnalysis}, una dintre bibliotecile
de integrare date ca exemplu, toate exemplele de analizoare și toate
testele automate au codul sursă scris în acest limbaj.

Înafară de preferință personală, argumentul principal pentru folosirea
limbajului C++ este interoperabilitatea bună cu limbajul C. Astfel, a
fost ușor de refolosit cod din biblioteca de înregistrare de evenimente
în programul SyncAnalysis pentru părțile de deserializare a fișierului
de evenimente, în timp ce restul programului și analizorii au putut fi
scriși într-un limbaj modern și foarte eficient. Biblioteca standard C++
este folosită peste tot în implementarea programului și în implementarea
analizorilor. Printre modulele folosite cel mai des se numără
\lstinline{<vector>}, \lstinline{<map>} și \lstinline{<string>}.

\subsection{Compilatorul Clang}

Limbajele C și C++, folosite în acest proiect, sunt limbaje a căror cod
sursă se compilează în cod nativ, executabil direct pe hardware. Pentru
a efectua această transformare în proiectul curent, am folosit exclusiv
compilatorul Clang\cite{Clang}, pentru portabilitatea acestuia între Mac
OS și Linux, viteza de compilare și integrarea elegantă cu alte unelte
de care am avut nevoie în dezvoltare cum ar fi
Clang-Tidy\cite{ClangTidy} și ThreadSanitizer\cite{ThreadSanitizer}.

\subsection{Uneltele de dezvoltare CMake și CTest}

Pentru proiecte mari de C++, invocarea manuală a compilatorului pentru
a recompila fiecare fișier modificat devine repede impractică.
CMake\cite{CMake} este un \textit{build system}: prin scrierea unui
fișier concis de configurare ce descrie structura fișierelor de cod
sursă din proiect, CMake automatizează recompilarea tuturor fișierelor
impactate de o schimbare, ducând astfel la o experiență plăcută de
dezvoltare a proiectului.

CTest este o componentă a proiectului CMake folosită pentru rularea
automată a testelor. Din nou, cu creșterea numărului de teste ce
trebuie efectuate după o schimbare în cod, devine impractică invocarea
manuală a acestora. CTest ajută la automatizarea acestui procedeu, din
nou spre a facilita o experiență plăcută de dezvoltare.

\subsection{Biblioteca pthread}

Biblioteca \lstinline{pthread}\cite{pthread} este interfața pentru
crearea și manipularea mai multor fire de execuție într-un proces. Este
parte din specificația standard POSIX pentru o interfață portabilă a
sistemelor de operare.

Biblioteca \lstinline{pthread} este folosită în ambele bibliotecile de
integrare, una din acestea având ca scop chiar simularea unei părți din
interfața publică oferită în antetul bibliotecii,
\lstinline{<pthread.h>}.

De asemenea, \lstinline{pthread} este folosită și în implementarea
bibliotecii de capturare de evenimente pentru crearea unui fir de
execuție separat care serializează evenimentele capturate în fișierul de
evenimente.

\subsection{Biblioteca DL}

Biblioteca \lstinline{DL}\cite{DL} este interfața descrisă în
specificația standard POSIX pentru găsirea, încărcarea și utilizarea
bibliotecilor dinamice. Această bibliotecă este folosită în proiect în
programul independent \lstinline{SyncAnalysis} pentru a încărca
analizorii specificați de client.

\subsection{Biblioteca libunwind}

Biblioteca \lstinline{libunwind}\cite{libunwind} este o facilitate
portabilă și eficientă pentru a observa stiva de apeluri de funcții în
timpul execuției unui program.

Biblioteca este folosită pentru a înregistra stiva de apeluri de funcții
când se capturează un eveniment. Folosirea bibliotecii acesteia este
preferabilă altor variante mai convenabile ce oferă direct simbolurile
asociate funcțiilor din stivă pentru că scopul bibliotecii de capturare
de evenimente este să fie cât mai eficientă, iar obținerea acestor
simboluri se poate face din programul \lstinline{SyncAnalysis}, după
terminarea programului client, doar pentru stivele de execuție necesare
în rapoarte.

\subsection{Programele addr2line, respectiv atos}

După cum s-a menționat și în secțiunea pentru \lstinline{libunwind},
biblioteca de capturare a evenimentelor serializează doar lista de
adrese din stiva de execuție în datele evenimentului în timpul execuției
programului client. Programul \lstinline{SyncAnalysis} transformă apoi
respectivele adrese în simboluri citibile de către client (nume de
funcții sau fișier și linie în codul sursă unde este făcut apelul),
pentru a fi atașate rapoartelor.

Pentru multe dintre sistemele de operare ce respectă standardul POSIX,
problema este rezolvată de programul \lstinline{addr2line} oferit de GNU
ca parte a proiectului \lstinline{binutils}\cite{binutils}. Programul
primește o listă de adrese de memorie ce reprezintă stiva de apeluri de
funcții și executabilul din care acestea au provenit și afișează exact
lista de simboluri asociate, și, dacă se pot afla, numele fișierului și
linia unde se află codul sursă pentru adresa respectivă din program.
Pentru că proiectul a fost dezvoltat în mare parte pe un calculator cu
sistemul de operare Mac OS (cu kernel-ul Darwin) unde programul
\lstinline{addr2line} nu dă rezultate satisfăcătoare, când se rulează
programul \lstinline{SyncAnalysis} pe Mac OS, acesta folosește
\lstinline{atos}\cite{atos} pentru a îndeplini aceleași sarcini.

\subsection{Biblioteca mcga-cli}

Programul \lstinline{SyncAnalysis} acceptă multe argumente de
configurare în linia de comandă (de exemplu directoarele în care să
caute analizori, reguli de includere și excludere pentru ce analizori
dintre cei găsiți să fie folosiți, dacă să afișeze sau nu informații
auxiliare pentru găsirea de erori în interiorul executabilului în sine
și altele).

Astfel, argumentele din linia de comandă sunt citite și interpretate
folosind o bibliotecă externă, \lstinline{mcga-cli}\cite{mcga-cli}.
Această bibliotecă a fost dezvoltată tot de autorul proiectului
\lstinline{SyncAnalysis} descris aici, dar este un proiect independent
de acesta.

\subsection{Git și GitHub}

Git\cite{git} este un sistem de versionare și menținere a istoricului de
modificări. A fost folosit în dezvoltarea acestui proiect exact cu acest
scop: de a menține un istoric central al tuturor modificărilor făcute
asupra codului sursă al proiectului și al acestui document.

GitHub\cite{GitHub} este o platformă comercială ce oferă un server de
git și o interfață web pentru a putea naviga ușor prin istoric.
\textit{Repository}-ul de git unde se găsește codul sursă al acestui
proiect este salvat pe serverul de git oferit de GitHub, și este
accesibil la adresa \url{https://github.com/darius98/sync-analysis}.

\subsection{Biblioteca Google Benchmark}

Deoarece o metrică importantă a bibliotecii de capturare de evenimente
este performanța, proiectul include și un tabel de măsurători făcute
pentru a estima latența și debitul acestei biblioteci (măsurătorile se
găsesc în secțiunea \textbf{\ref{library-performance}}). Pentru
efectuarea acestor măsurători a fost folosită biblioteca Google
Benchmark\cite{GoogleBenchmark}, un proiect \textit{open-source} ce are
ca scop exact asta: efectuarea de măsurători exacte de performanță cu
efort minim din partea programatorului.
