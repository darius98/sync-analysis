\section{Concluzii}

Complexitatea aplicațiilor software crește constant, și pentru ca
dezvoltatorii de aplicații să poată păstra același nivel de calitate a
produselor în ciuda cerințelor din ce în ce mai dificile, au nevoie de
unelte din ce în ce mai puternice care să îi ajute. Dar dezvoltarea
acestor unelte, mai ales a celor \textit{specifice} unei aplicații,
necesită de asemenea efort și timp, de multe ori mai mult decât sunt
dispuse companiile și dezvolatorii să investească.

Proiectul prezentat în această lucrare și-a îndeplinit cu succes scopul
de a simplifica procedeul de dezvoltare a acestor unelte. Evenimente
de interes sunt capturate eficient și automat în timpul execuției
aplicației, necesitând schimbări minime (sau uneori deloc) în codul
sursă al aplicațiilor analizate. Analizori \textit{post-mortem} ai
acestor evenimente, cu aplicații practice reale, pot fi compuși și
folosiți cu ușurință, atât pentru analiza primitivelor de sincronizare
cât și pentru orice alt domeniu, uneori necesitând mai puțin de 100 de
linii de cod.

Industria software are nevoie de unelte de dezvoltare, iar proiectul
prezentat este un pas spre facilitarea compunerii acestora.
