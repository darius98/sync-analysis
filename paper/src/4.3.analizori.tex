\subsection{Analizori}\label{analyzers}

După execuția programului client integrat cu biblioteca pentru capturare
de evenimente și una din bibliotecile de mai sus, evenimentele capturate
trebuie acum analizate folosind programul \lstinline{SyncAnalysis}. Însă
după cum am mai spus, acest program nu face pe cont propriu nicio
analiză, ci se bazează pe \textit{analizorii} încărcați să emită
diagnosticele în sine, programul doar ajutând acești analizori cu munca
de rutină. Astfel, ca parte dintr-o unealtă de analiză trebuie compusă
și o suită de analizori care să facă propriu-zis analiza.

Vom descrie în continuare un exemplu de astfel de suită: 4 analizori ce
au ca scop observarea folosirii \textit{excesive} de primitive de
sincronizare.

\subsubsection{useless-lock}

Observație: dacă un \textit{mutex}, \textit{mutex recursiv} sau un
\textit{read-write lock} este \textit{locked} de un singur fir de
execuție, primitiva respectivă nu este necesară programului.

Pe baza observației de mai sus se poate concepe un simplu algoritm care
analizează apariția acestei situații. Pentru fiecare eveniment care
marchează crearea unui \textit{lock} (oricare din primitivele de mai
sus), se crează o intrare pentru respectivul lock într-un tabel. Apoi,
pentru orice eveniment capturat cu câmpul \lstinline{object_id} egal cu
obiectul inițial se înregistrează câmpul \lstinline{thread_id} al
evenimentului în intrarea pentru obiectul respectiv în tabel. Dacă un
obiect ajunge să aibă 2 fire de execuție diferite stocate în intrarea
sa din tabel, acesta poate fi șters din tabel, deoarece nu mai prezintă
interes. Dacă atunci când se capturează evenimentul de distrugere a
\textit{lock}-ului (sau când se termină lista de evenimente) intrarea
acestuia încă se află în tabel, este emis un diagnostic cu nivelul de
severitate de \textit{avertisment}, pentru că respectivul \textit{lock}
nu este neapărat necesar.

Implementarea acestui analizor conține un total de 89 de linii de cod
C++.

\subsubsection{lock-shadow}

Implementarea acestui analizor conține un total de 103 linii de cod
C++.

\subsubsection{redundant-recursive-mutex}

Implementarea acestui analizor conține un total de 73 de linii de cod
C++.

\subsubsection{redundant-rwlock}

Implementarea acestui analizor conține un total de 76 de linii de cod
C++.
