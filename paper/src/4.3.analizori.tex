\subsection{Analizori}\label{analyzers}

După execuția programului client integrat cu biblioteca pentru capturare
de evenimente și una din bibliotecile de mai sus, evenimentele capturate
trebuie acum analizate folosind programul \lstinline{SyncAnalysis}. Însă
după cum am mai spus, acest program nu face pe cont propriu nicio
analiză, ci se bazează pe \textit{analizorii} încărcați să emită
diagnosticele în sine, programul doar ajutând acești analizori cu munca
de rutină. Astfel, ca parte dintr-o unealtă de analiză trebuie compusă
și o suită de analizori care să facă propriu-zis analiza.

Vom descrie în continuare un exemplu de astfel de suită: 4 analizori ce
au ca scop observarea folosirii \textit{excesive} de primitive de
sincronizare.

Vom folosi în explicațiile ce urmează notațiile:

$t(e)=$ momentul de timp la care a avut loc evenimentul $e$

$F(e)=$ identitatea firului de execuție pe care a fost capturat
evenimentul $e$

$LU(a)=$ mulțimea tuturor evenimentelor capturate cu $a$ ca obiect țintă
(unde $a$ poate fi un \textit{mutex}, un \textit{mutex recursiv} sau un
\textit{read write lock}), pentru care tipul evenimentului este unul de
\textit{lock}, \textit{unlock}, \textit{read lock}, \textit{write lock},
\textit{read unlock} sau \textit{write unlock}.

\subsubsection{useless-lock}

Observație: pentru un obiect $a$, fie
$S_a = \{\ F\ |\ \exists e \in LU(a) \ a.i.\ F(e) = F\ \}$. Dacă
$|S_a| \leq 1$, atunci obiectul $a$ nu este necesar corectitudinii
programului.

Se poate concepe un simplu algoritm care analizează apariția situației
descrise în observație. Pentru fiecare eveniment care marchează crearea
unui obiect $a$, se crează o intrare cu o mulțime vidă pentru $a$
într-un tabel. Apoi, pentru orice eveniment capturat $e$ cu obiectul
țintă $a$ se adaugă $F(e)$ în intrarea pentru $a$ din tabel. Dacă un
obiect ajunge să aibă cardinalul mulțimii din intrarea sa din tabel
minim 2, acesta poate fi șters din tabel, deoarece nu mai prezintă
interes. Dacă atunci când se capturează evenimentul de distrugere a
obiectului (sau când se termină lista de evenimente) intrarea acestuia
încă se află în tabel, este emis un diagnostic cu nivelul de severitate
de \textit{avertisment}, pentru că respectivul obiect nu este necesar
corectitudinii programului.

Acest analizor este implementat în 89 de linii de cod C++.

\subsubsection{lock-shadow}

Observație: pentru un obiect $a$ fixat, dacă $\exists$ un
obiect $b\ \neq a$ astfel încât $\forall\ L_a,\ U_a$ evenimente de
\textit{lock} respectiv \textit{unlock} consecutive cu obiectul țintă
$a$, $\exists\ L_b,\ U_b$ evenimente de \textit{lock} respectiv
\textit{unlock} consecutive pe obiectul $b$ astfel încât 
$t(L_b) < t(L_a) < t(U_a) < t(U_b)$ și $F(L_a) = F(L_b)$, atunci
obiectul $a$ nu este necesar corectitudinii programului.

Implementarea acestui analizor este puțin mai complexă decât a celui de
mai sus. S-au folosit două structuri de date diferite: un tabel $T$ ce 
menține lista de potențiale obiecte $b$ pentru fiecare obiect $a$ în
parte (folosind notațiile din observația de mai sus) și un alt tabel $S$
ce menține lista de obiecte \textit{locked} de fiecare fir de execuție
în parte. Astfel, pentru fiecare eveniment care marchează crearea unui
obiect $a$ se adaugă o intrare în prima structură de date, considerând
că mulțimea de posibile obiecte $b$ pentru obiectul $a$ este mulțimea
tuturor obiectelor (în implementare se folosește valoarea
\lstinline{NULL} pentru a reprezenta mulțimea infinită):
$T(a) \gets $ \lstinline{NULL}. La momentul unui eveniment de
\textit{lock} $L$ asupra obiectului $a$, mai întâi se schimbă mulțimea
de potențiale obiecte $b$ a lui $a$  prin intersectarea acesteia cu
mulțimea obiectelor \textit{locked} de firul de execuție a lui $L$:
$T(a) \gets T(a) \cap S(F(L))$. Apoi obiectul $a$ este adăugat în
mulțimea de obiecte \textit{locked} de firul de execuție al obiectului
$L$: $S(F(L)) \gets S(F(L)) \cup \{a\}$. Pentru un eveniment de tip
\textit{unlock} $U$ pe un obiect $a$, acesta este scos din mulțimea de
obiecte \textit{locked} a firului de execuție $F(U)$:
$S(F(U)) \gets S(F(U)) \backslash \{a\}$.

Când este procesat un eveniment pentru distrugerea unui obiect $a$ (sau
când procesarea evenimentelor se încheie) și $T(a) \neq \emptyset$,
fiecare obiect din $T(a)$ respectă proprietatea menționată pentru
obiectul $b$ în observația de mai sus. Astfel, analizorul emite un
diagnostic cu nivelul de \textit{avertisment} pentru fiecare astfel de
pereche $(a, b)$ cu $b \in T(a)$.

Acest analizor este implementat în 103 linii de cod C++.

\subsubsection{redundant-recursive-mutex}

Observație: Dacă un \textit{mutex recursiv} nu este niciodată
\textit{locked} de două ori la rând (fără un \textit{unlock} între ele),
acesta poate fi înlocuit cu un \textit{mutex} simplu.

Pentru a implementa acest analizor, menținem tabelul $S$ la fel ca
pentru analizorul \textit{lock-shadow} descris mai sus, dar cu intrări
doar pentru obiectele de tipul \textit{mutex recursiv} și lista $L$ de
obiecte de tipul \textit{mutex recursiv} care sunt potențial redundante.
Pentru fiecare eveniment care marchează crearea unui astfel de obiect,
obiectul este adăugat în lista $L$. Pentru fiecare eveniment $e$ de
tipul \textit{lock} pe un astfel de obiect $a$, se verifică mai întâi
dacă $a$ se află în $S(F(e))$, și dacă da, acesta este scos din lista
$L$ (dacă încă era), deoarece a fost făcut un \textit{lock recursiv} pe
obiectul $a$.

Când este procesat un eveniment pentru distrugerea unui obiect $a$ (sau
când procesarea evenimentelor se încheie), dacă $a \in L$, atunci
analizorul emite un diagnostic cu nivelul de \textit{avertisment} pentru
obiectul $a$, deoarece acesta nu a fost niciodată folosit în mod
recursiv și ar putea fi înlocuit cu un \textit{mutex} simplu. În
practică implementarea este puțin mai complexă pentru a evita emiterea
unui diagnostic pentru obiecte ce nu au fost niciodată \textit{locked},
deoarece analizorul \textit{useless-lock} va emite oricum diagnostic
pentru această situație.

Acest analizor este implementat în 73 de linii de cod C++.

\subsubsection{redundant-rwlock}

Observație: Dacă un \textit{read write lock} nu este niciodată
\textit{read-locked}, acesta poate fi înlocuit cu un \textit{mutex}
simplu. De asemenea, dacă un \textit{read write lock} nu este niciodată
\textit{write-locked}, acesta nu este defapt necesar programului
(deoarece nu blochează defapt niciun acces concurent).

Implementarea acestui analizor este chiar mai ușoară decât a celor de
mai sus: se menține o singură structură de date de tip tabel, $S$, în
care se crează câte o intrare pentru un obiect de tipul
\textit{read write lock} în momentul procesării evenimentelor de tipul
\lstinline{SA_EV_RWLOCK_ON_CREATE}. În intrarea din tabel sunt menținuți
doi indicatori \lstinline{bool} pentru respectivul obiect:
\lstinline{was_ever_rd_locked} care este marcat atunci
când este procesat un eveniment \lstinline{SA_EV_RWLOCK_AFTER_RD_LOCK}
cu acest obiect ca țintă și \lstinline{was_ever_wr_locked} care este
marcat atunci când este procesat un eveniment de tipul
\lstinline{SA_EV_RWLOCK_AFTER_WR_LOCK} cu acest obiect ca țintă.

Când este procesat un eveniment pentru distrugerea unui obiect (sau când
procesarea evenimentelor se încheie), analizorul verifică cei doi
indicatori din tabel. Dacă indicatorul \lstinline{was_ever_rd_locked}
a fost marcat, dar \lstinline{was_ever_wr_locked} nu, atunci se emite un
diagnostic cu nivelul de \textit{avertisment} pentru că obiectul
respectiv nu este necesar. Dacă indicatorul
\lstinline{was_ever_wr_locked} a fost marcat, dar
\lstinline{was_ever_rd_locked} nu, atunci se emite un diagnostic cu
nivelul de \textit{avertisment} pentru că obiectul respectiv poate fi
înlocuit cu un simplu \textit{mutex}. Analizorul evită astfel să emită
diagnostice pentru obiectele ce nu au fost niciodată \textit{locked} (în
niciun fel), deoarece analizorul \textit{useless-lock} va emite oricum
diagnostic pentru această situație.

Acest analizor este implementat în 76 de linii de cod C++.
