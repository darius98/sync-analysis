\subsection{Instalarea și compilarea proiectului}
Proiectul poate fi descărcat, instalat și folosit pe orice calculator cu
sistemul de operare o distribuție de Linux, OpenBSD sau Mac OS și acces
la internet. De asemenea, calculatorul trebuie să aibă instalate
următoarele programe:
\begin{itemize}
    \item \lstinline{as, ar, ld, addr2line} -- de obicei aceste programe
    sau alte programe echivalente vin preinstalate cu orice distribuție
    de Linux, Mac OS, FreeBSD sau OpenBSD. Ele fac parte din colecția de
    utilitare \lstinline{binutils}\cite{binutils} și sunt disponibile
    gratuit pentru descărcare pe website-ul oficial
    GNU\cite{GNUWebsite}.
    \item \lstinline{CMake}\cite{CMake} -- versiunea \lstinline{3.15}
    sau mai nouă
    \item \lstinline{make} -- orice versiune compatibilă cu versiunea de
    CMake instalată
    \item un compilator de C și C++ care suportă standardele C11,
    respectiv C++17 în întregime (de exemplu \lstinline{Clang 10}
    \item \lstinline{git}\cite{git} -- minim versiunea \lstinline{2.0}
\end{itemize}

Proiectul este disponibil pe internet, stocat pe server-ul GitHub.
Astfel, acesta poate fi descărcat prin următoarea comandă
\lstinline{shell}:

\begin{minipage}{\linewidth}
\begin{lstlisting}
    % git clone https://github.com/darius98/sync-analysis.git
\end{lstlisting}
\end{minipage}

Deoarece \textit{repository}-ul de git include biblioteca
\lstinline{mcga-cli}\cite{mcga-cli} ca submodul de git, trebuie
inițializată logica pentru submodule de git și descărcat separat acest
submodul:

\begin{minipage}{\linewidth}
\begin{lstlisting}
    % cd sync-analysis/
    % git submodule update --init
\end{lstlisting}
\end{minipage}

Odată ce \textit{repository}-ul principal și \textit{submodul}-ul
\lstinline{mcga-cli} au fost descărcate cu succes, proiectul poate fi
compilat prin următoarea secvență de comenzi, executate în directorul de
bază al proiectului:

\begin{minipage}{\linewidth}
\begin{lstlisting}
    % cmake .
    % make all
\end{lstlisting}
\end{minipage}

După compilarea proiectului, acesta poate fi instalat în sistem folosind
următoarea comandă, executată tot în directorul de bază al proiectului:

\begin{minipage}{\linewidth}
\begin{lstlisting}
    % make install
\end{lstlisting}
\end{minipage}

Această comandă adaugă fișiere noi în directoarele de sistem. De obicei,
aceste fișiere se adaugă în directorul \lstinline{/usr/local/}, deci pe
multe sisteme această comandă va avea nevoie de permisiuni de
administrator. Soluția, de exemplu pe sistemul de operare Ubuntu, este
de a rula comanda folosind \lstinline{sudo}
(\lstinline{sudo make install}). Fișierele ce se adaugă în sistem prin
instalare sunt următoarele:

\begin{itemize}
    \item biblioteca pentru capturare de evenimente:
    \lstinline{lib/libsync_analysis.so} (pentru Mac OS extensia este
    \lstinline{.dylib}) și interfața C a acesteia:
    \lstinline{include/sync_analysis.h}
    \item biblioteca pentru integrare \lstinline{lib/libcxxsync.a} și
    interfața C++ a acesteia: directorul \lstinline{include/cxxsync/}
    \item biblioteca pentru integrare
    \lstinline{lib/libsyan_pthread_shim.so} (pentru Mac OS extensia este
    \lstinline{.dylib})
    \item biblioteca pentru integrare
    \lstinline{lib/libsyan_stdcxx_shim.so} (pentru Mac OS extensia
    este \lstinline{.dylib}) și interfața C++ a acesteia: directorul
    \lstinline{include/stdcxx_shim/}
    \item programul independent \lstinline{SyncAnalysis}
    \lstinline{bin/sync_analysis}
    \item interfața C++ pentru dezvoltarea analizorilor
    \lstinline{include/syan_analyzer_api/}
    \item analizorii preinstalați directorul \lstinline{syan-analyzers/}
\end{itemize}
