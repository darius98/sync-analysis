\subsection{Instalarea și compilarea proiectului}
Proiectul poate fi descărcat, instalat și folosit pe orice calculator cu
sistemul de operare o distribuție de GNU/Linux sau Mac OS și cu acces
la internet. De asemenea, calculatorul trebuie să aibă instalate
următoarele programe:
\begin{itemize}
    \item \lstinline{as, ar, ld, addr2line} -- de obicei aceste programe
    vin preinstalate cu orice distribuție de GNU/Linux sau Mac OS. Ele
    fac parte din colecția de utilitare
    \lstinline{binutils}\cite{binutils} și sunt disponibile gratuit
    pentru descărcare pe website-ul oficial GNU\cite{GNUWebsite} în
    cazul în care nu sunt deja instalate.
    \item \lstinline{CMake}\cite{CMake} -- versiunea \lstinline{3.15}
    sau mai nouă
    \item \lstinline{make} -- orice versiune compatibilă cu versiunea de
    CMake instalată
    \item \lstinline{atos}\cite{atos} -- doar pentru Mac OS, unde este
    mereu preinstalat
    \item un compilator de C și C++ care suportă standardele C11,
    respectiv C++17 în întregime (de exemplu
    \lstinline{Clang10}\cite{Clang})
    \item \lstinline{git}\cite{git} -- minim versiunea \lstinline{2.0}
\end{itemize}

Proiectul este disponibil pe internet, stocat pe server-ul GitHub.
Astfel, acesta poate fi descărcat și instalat prin următoarea secvență
de comenzi \lstinline{shell}:

\begin{lstlisting}[language=bash,
                   numbers=left,
                   xleftmargin=3em,
                   caption=Instrucțiuni pentru instalarea proiectului]
git clone https://github.com/darius98/sync-analysis.git
cd sync-analysis/
git submodule update --init
cmake .
make all
make install
\end{lstlisting}

Comanda 1 descarcă \textit{repository}-ul de git al proiectului de pe
server-ul GitHub. Acesta conține tot codul sursă asociat proiectului.
Toate comenzile mai departe trebuie executate în directorul de bază al
proiectului, așa că comanda 2 schimbă directorul curent pentru restul
comenzilor.

Deoarece \textit{repository}-ul include biblioteca
\lstinline{mcga-cli}\cite{mcga-cli} ca submodul, aceasta trebuie
descărcată separat după inițializarea logicii de git pentru submodule.
Comanda 3 execută ambele acțiuni.

Odată ce \textit{repository}-ul principal și submodulul
\lstinline{mcga-cli} au fost descărcate cu succes, proiectul poate fi
compilat prin comenzile 4-5. După compilarea proiectului, acesta poate
fi instalat în sistem folosind comanda 6. Această comandă adaugă fișiere
noi în directoarele de sistem. De obicei, aceste fișiere se adaugă în
directorul \lstinline{/usr/local/}, deci pe multe sisteme această
comandă va avea nevoie de permisiuni de administrator. Soluția, de
exemplu pe sistemul de operare Ubuntu, este de a rula comanda folosind
\lstinline{sudo} (\lstinline{sudo make install}). Fișierele ce se adaugă
în sistem prin instalare sunt următoarele:

\begin{itemize}
    \item biblioteca pentru capturare de evenimente:
    \lstinline{lib/libsync_analysis.so} (pe Mac OS extensia este
    \lstinline{.dylib}) și interfața C a acesteia în fișierul de
    interfață \lstinline{include/sync_analysis.h}
    \item programul independent \lstinline{SyncAnalysis} în fișierul
    \lstinline{bin/sync_analysis}
    \item interfața C++ pentru dezvoltarea analizorilor
    \lstinline{include/syan_analyzer_api/}
    \item biblioteca pentru integrare \lstinline{lib/libcxxsync.so}
    (respectiv \lstinline{.dylib} pe Mac OS) și interfața C++ a
    acesteia în directorul \lstinline{include/cxxsync/}
    \item biblioteca pentru integrare
    \lstinline{lib/libsyan_pthread_shim.so} (pentru Mac OS extensia este
    \lstinline{.dylib})
    \item analizorii dați ca exemplu, în directorul
    \lstinline{lib/syan-analyzers/}
\end{itemize}
